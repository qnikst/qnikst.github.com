-----
title: Лекция 1
date: unknown
author: text by Vershilo A.B. <br /> lecture by Tsiganov A. V.
mathjax: yes
-----
В классическом случае рассматриваем голономные системы со связями.

Связи бывают двух видов

\begin{enumerate}
  \item не удерживающие, одностороннее взаимодействие
    $$ f(r_1,\ldots,r_n,t)\geq 0 $$
  \item удерживающие, двустороннее взаимодействие
    $$ f(r_1,\ldots,r_n,t)=0$$
\end{enumerate}

\begin{definition}
Если в уравнения входит скорости - то система называется неголономной:
\end{definition}

Примеры:
\begin{itemize}
  \item tippletop
  \item celtic stone
\end{itemize}

пример "конёк" (TODO график)

Рассмотрим голономные скалярные связи на примере двойного маятника:
(TODO картинка)
\begin{equation}
  f(\vec{r_1},\ldots,\vec{r_n}) = 0
\end{equation}

Конфигурационное пространство:
\begin{equation}
  0 \leq \phi \leq 2\pi,~~~ 0 \leq \theta \leq 2\pi
\end{equation}

Это тор (TODO картинка)

Итого возникает вопрос: как квантовать тор?

Уравнение Лагранжа второго рода
\begin{equation}
\dfrac{d}{dt} \frac{\partial L}{\partial \dot{q}_i} - \frac{\partial L}{\partial q_i} = 0, i=1\ldots m
\end{equation}

\begin{equation}
L(q_1,\ldots,q_n,\dot{q}_1,\ldots,\dot{q}_n,t)
\end{equation}

Осуществляем переход: $(q,\dot{q}) \rightarrow (q,p), ~~~ p_i = \frac{\partial L}{\partial \dot{q}_i}$

Получаем уравнения Гамильтона ($n$-шт)

\begin{equation}
    H = \sum p_i \dot{q}_i - l 
\end{equation}

Откуда получаем уравнения второго рода ($2n$ штук)

\begin{equation}
    \left\{
        \begin{align}
              \dot{q}_i = \fraq{\partial H}{\partial p_i} \\
              \dot{p}_i = - \fraq{\partial H}{\partial \dot{q}_i
        \end{align}
    \right.
\end{equation}

Для кватнования у $L$ и $H$ разные теории и рассматривать $L$ мы не будем.

$$H=H(p,q,\not t)$$


\begin{equation}
    \frac{dH}{dt} = \sum_{i=1}^m \frac{\partial H}{\partial p}\dot{p}_i + \sum \frac{\partial H}{\partial q_i} \dot{q}_i = 0
\end{equation}

в этом случае $H$ сохраняется и является первым интегралом движения.

\begin{definition}[Первый интеграл]
    $F$ - называется интегралом $\Leftrightarrow$ $\dfrac{dF}{dt} (p,q) = 0$
\end{definition}

Интегралы позволяют уменьшть количество уравнений движения. И каждый интеграл связан с симметриями.

\begin{definition}
    Скобками Пуассона $\{p,q\}$ называют:
    \begin{equation}
        \left\{f,g\right\} = \sum \frac{\partial f}{\partial q_i} \cdot \frac{\partial g}{\partial p_i} - 
                                  \frac{\partial f}{\partial p_i} \cdot \frac{\partial g}{\partial q_i} 
    \end{equation}
\end{definition}

Если скобки Пуассона  совпадают с алгеброй Ли, то они называется скобками Ли-Пуассона.

Пусть $\mathcal M$ многообразие с локальными координатами $z_1,\ldots,z_m$

$f(z_1,\ldots,z_m)$ - функция на многообразии, то мы молжем взять внешнюю производную

\begin{equation}
    df = \left( \begin{align}
                  \frac{\partial f}{\partial z_1}\\
                  \vdots\\
                  \frac{\partial f}{\partial z_n}\\
                \end{align}
         \right)
\end{equation}

тогда скобки Пуассона которые задают многообразие равны:

$$ \{f,g\} = \langle df,Pdg\rangle$$

В нашем случае:
\begin{equation}
P = \left(\begin{matrix} 0 & I \\ -I & 0\end{matrix}\right)
\end{equation}

би-вектор Пуассона, структурный тензор алгебры Ли.


Пусть динамика задается Гамильтонианом $H$. С этим гамильтонианом
мы можем связать векторное поле.

$$X_H = P dH \in T\mathcal M$$

$$X_H(x) = \{H,x\}$$

Запишем динамику эволюции. Вводим время $t$

(TODO картинку)

\begin{definition}
  эволюция гамильтоновой системы

  $(t,f) \rightarrow f_t$ - значение во время $t$, $f\in C^\infty(\mathcal M)$
\end{definition}

Фазовый поток:

$f(q_1,\ldots,q_m,p_1,\ldots,p_m,t) \rightarrow f(q_1(t),\ldots,q_m(t),p_1(t),\ldots,p_m(t))$ 

Если время $t$ - малое то:

$$f(t) = exp(tH) \circ f = \sum \frac{t^n}{n!} \{\underbrace{f\cdot f\cdot \ldots \cdot f}_{n},H\}$$

Уравнения движения принимают вид $X_H(x) = \{H,x\} = P dH(x)$.

Разберемся сначала:

\begin{equation}
     i \hbar \frac{\partial \Psi}{\partial t} = \hat H \Phi
\end{equation}

$$H = \frac{p^2}{2m} + V(q), p = \frac{i\hbar \partial}{\partial q}$$

Наша задача написать операторы.

В квадратичном случае $\Psi=e^{i\hbar R}\Psi \rightarrow  \hat H \Psi = E \Psi$. тут надо 
найти все векторы $\Psi_1,\ldots,\Psi_k$.

Задача:
\begin{itemize}
    \item зная Лагранжево подмножество, как построить собственную фунцию /Шон Бетье лекции по геометрическоу квантованию/
    \item алгебраическое квантование
        есть $\{f,g\}$ и считаем, что $\{f,g\} = \lim_{h\rightarrow 0} \frac{[f,g]}{h}$ \\
        $\{f,g\} = i h [~]_0 + h^2 [~]_1$, где $[~]$ - граница когомологии Пуассона-Линхеровича, деформационное кватнование.
\end{itemize}


Теорема Лиувиля $\mathcal M$, $dim \mathcal M=2n$, симлектическое многообразие
(т.е. существует симлектическая форма $\omega = P^{-1} \Rightarrow \exists \{\cdot,\cdot\}$)
\begin{enumerate}
    \item есть $m$ ингтегралов движения $\{H_1,\ldots,H_m\}$, $\frac{dH_i}{dt}=0$
    \item интегралы функционально независимы
    \item интегралы находятся в инволюции $\{H_i,H_j\} = 0$
\end{enumerate} 
Тогда такие системы называются интегрируемыми в квадратурах (по Лиувилю)


Арнольд (1960-е)

Многообразие это объединение торов и мы можем перейти в окрестности тора в координаты (действие-угол)

действие перпендикулярно тору, т.е. переводит с тора на другой, угол описывает вращение вокруг тора.

С точки зрения кватновой механики система интегрируема или близка к интегрируемой $\Leftrightarrow$
(TODO картинку)

Барабан Сеная (TODO картинку)

Простейшее определение

\begin{itemize}
  \item $\hat H_1, \hat H_2,\ldots, \hat H_n$
  \item $[\hat H_i, \hat H_j] = 0$
  \item $H$ - независимы
\end{itemize}

тогда кватновая система интегрируема

Возникают вопросы:
\begin{enumeration}
  \item что такое $m$
  \item что такое коммутатор
  \item что такое независимость
\end{enumeration}


%$$[H_1,H_2] = 0$$
%$$H\Psi^{(1)}_k = \lambda_k^{(1)} \Psi_k^{(1)}$$
%$$H_1(H_2\Psi_k) = H_2H_1\Psi_k^{(1)} = \lambda(H_2 \Psi^{(1)}_k$$

В теории рассеяния интегрируемые системы $\Leftrightarrow$ системы без дифракции

В $\partial\Omega$ гран многообразий
(TODO картинки)


